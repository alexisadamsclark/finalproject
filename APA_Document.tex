\documentclass[man]{apa6}
\usepackage{lmodern}
\usepackage{amssymb,amsmath}
\usepackage{ifxetex,ifluatex}
\usepackage{fixltx2e} % provides \textsubscript
\ifnum 0\ifxetex 1\fi\ifluatex 1\fi=0 % if pdftex
  \usepackage[T1]{fontenc}
  \usepackage[utf8]{inputenc}
\else % if luatex or xelatex
  \ifxetex
    \usepackage{mathspec}
  \else
    \usepackage{fontspec}
  \fi
  \defaultfontfeatures{Ligatures=TeX,Scale=MatchLowercase}
\fi
% use upquote if available, for straight quotes in verbatim environments
\IfFileExists{upquote.sty}{\usepackage{upquote}}{}
% use microtype if available
\IfFileExists{microtype.sty}{%
\usepackage{microtype}
\UseMicrotypeSet[protrusion]{basicmath} % disable protrusion for tt fonts
}{}
\usepackage{hyperref}
\hypersetup{unicode=true,
            pdftitle={Final Project: The Relationship between Sleep, Depression, Quality of Life, and Socioeconomic Status},
            pdfauthor={Alexis Adams-Clark, Andrew Fridman, \& Xi Yang},
            pdfkeywords={sleep, depression, quality of life, SES},
            pdfborder={0 0 0},
            breaklinks=true}
\urlstyle{same}  % don't use monospace font for urls
\usepackage{graphicx,grffile}
\makeatletter
\def\maxwidth{\ifdim\Gin@nat@width>\linewidth\linewidth\else\Gin@nat@width\fi}
\def\maxheight{\ifdim\Gin@nat@height>\textheight\textheight\else\Gin@nat@height\fi}
\makeatother
% Scale images if necessary, so that they will not overflow the page
% margins by default, and it is still possible to overwrite the defaults
% using explicit options in \includegraphics[width, height, ...]{}
\setkeys{Gin}{width=\maxwidth,height=\maxheight,keepaspectratio}
\IfFileExists{parskip.sty}{%
\usepackage{parskip}
}{% else
\setlength{\parindent}{0pt}
\setlength{\parskip}{6pt plus 2pt minus 1pt}
}
\setlength{\emergencystretch}{3em}  % prevent overfull lines
\providecommand{\tightlist}{%
  \setlength{\itemsep}{0pt}\setlength{\parskip}{0pt}}
\setcounter{secnumdepth}{0}
% Redefines (sub)paragraphs to behave more like sections
\ifx\paragraph\undefined\else
\let\oldparagraph\paragraph
\renewcommand{\paragraph}[1]{\oldparagraph{#1}\mbox{}}
\fi
\ifx\subparagraph\undefined\else
\let\oldsubparagraph\subparagraph
\renewcommand{\subparagraph}[1]{\oldsubparagraph{#1}\mbox{}}
\fi

%%% Use protect on footnotes to avoid problems with footnotes in titles
\let\rmarkdownfootnote\footnote%
\def\footnote{\protect\rmarkdownfootnote}


  \title{Final Project: The Relationship between Sleep, Depression, Quality of
Life, and Socioeconomic Status}
    \author{Alexis Adams-Clark\textsuperscript{1}, Andrew
Fridman\textsuperscript{1}, \& Xi Yang\textsuperscript{1}}
    \date{}
  
\shorttitle{Final Project}
\affiliation{
\vspace{0.5cm}
\textsuperscript{1} University of Oregon Department of Psychology}
\keywords{sleep, depression, quality of life, SES\newline\indent Word count: X}
\usepackage{csquotes}
\usepackage{upgreek}
\captionsetup{font=singlespacing,justification=justified}

\usepackage{longtable}
\usepackage{lscape}
\usepackage{multirow}
\usepackage{tabularx}
\usepackage[flushleft]{threeparttable}
\usepackage{threeparttablex}

\newenvironment{lltable}{\begin{landscape}\begin{center}\begin{ThreePartTable}}{\end{ThreePartTable}\end{center}\end{landscape}}

\makeatletter
\newcommand\LastLTentrywidth{1em}
\newlength\longtablewidth
\setlength{\longtablewidth}{1in}
\newcommand{\getlongtablewidth}{\begingroup \ifcsname LT@\roman{LT@tables}\endcsname \global\longtablewidth=0pt \renewcommand{\LT@entry}[2]{\global\advance\longtablewidth by ##2\relax\gdef\LastLTentrywidth{##2}}\@nameuse{LT@\roman{LT@tables}} \fi \endgroup}


\DeclareDelayedFloatFlavor{ThreePartTable}{table}
\DeclareDelayedFloatFlavor{lltable}{table}
\DeclareDelayedFloatFlavor*{longtable}{table}
\makeatletter
\renewcommand{\efloat@iwrite}[1]{\immediate\expandafter\protected@write\csname efloat@post#1\endcsname{}}
\makeatother

\authornote{ We would like to acknowledge Daniel Anderson for
introducing us to Papaja and thank our classmates in Introduction to
Data Science with R.

Correspondence concerning this article should be addressed to Alexis
Adams-Clark, 1585 E 13th Ave, Straub 339, Eugene, OR 97403. E-mail:
\href{mailto:aadamscl@uoregon.edu}{\nolinkurl{aadamscl@uoregon.edu}}}

\abstract{
One or two sentences providing a \textbf{basic introduction} to the
field, comprehensible to a scientist in any discipline.

Two to three sentences of \textbf{more detailed background},
comprehensible to scientists in related disciplines.

One sentence clearly stating the \textbf{general problem} being
addressed by this particular study.

One sentence summarizing the main result (with the words ``\textbf{here
we show}'' or their equivalent).

Two or three sentences explaining what the \textbf{main result} reveals
in direct comparison to what was thought to be the case previously, or
how the main result adds to previous knowledge.

One or two sentences to put the results into a more \textbf{general
context}.

Two or three sentences to provide a \textbf{broader perspective},
readily comprehensible to a scientist in any discipline.


}

\begin{document}
\maketitle

\section{\texorpdfstring{Methods
\footnote{blah blah blah}}{Methods }}\label{methods}

\subsection{Participants}\label{participants}

Participants were recruited through the University of Oregon Human
Subjects Pool consisting of undergraduate students enrolled in
introductory psychology and linguistics courses. Students received
academic credit in exchange for their participants. Participants were
not aware of the subject of the study before scheduling their
participation, so participants did not self-select into the study.
Although they could leave the study after reading informed consent, no
participants chose to do so.

In total, {[}\#{]} participants were included in this study, all of who
had complete data. {[}insert demographic characteristics{]}

\subsection{Materials}\label{materials}

\subsubsection{Depressive Symptoms
Scale}\label{depressive-symptoms-scale}

A scale to measure depressive symptoms was created by the researchers
for the purposes of this study. Participants were instructed to rate
each item on Likert-type scale, where 1 corresponds to \enquote{strongly
disagree,} and 5 corresponds to \enquote{strongly agree.} The scale
consisted of three items, including: \enquote{Over the last 2 weeks, I
have felt little interest or pleasure in doing things;} \enquote{Over
the last 2 weeks, I have felt down, depressed, or hopeless;} and
\enquote{Over the last 2 weeks, I have felt tired or had little energy.}
Scale items were summed to create a depression score, where higher
scores represent higher depressive symptoms.

\subsubsection{Quality of Life Scale}\label{quality-of-life-scale}

A scale to measure participants' quality of life was created by the
researchers. Participants were instructed to rate each item on
Likert-type scale, where 1 corresponds to \enquote{strongly disagree,}
and 5 corresponds to \enquote{strongly agree.} The scale consisted of
four items, including: \enquote{My life is ideal;} \enquote{I am
satisfied with my life;} \enquote{So far I have been able get the
important things I want from life;} and \enquote{I have accomplished
many of the things in my life.} Scale items were summed to create a
total quality of life score, where higher scores represent better
quality of life.

\subsubsection{Sleep Quality Scale}\label{sleep-quality-scale}

A scale to measure participants' sleep quality was created by the
researchers. Participants were instructed to rate each item on
Likert-type scale, where 1 corresponds to \enquote{strongly disagree,}
and 5 corresponds to \enquote{strongly agree.} The scale consisted of
three items, including: \enquote{I am satisfied with the time I spend
sleeping;} \enquote{I am satisfied with my quality of sleep}; and
\enquote{When I wake up, I feel refreshed.} Scale items were summed to
create a total sleep quality score, where higher scores represent better
sleep quality.

\subsubsection{Demographics
Questionnaire}\label{demographics-questionnaire}

Participants were also asked to complete a demongraphics quationnaire,
which asked about their socioeconomic status, race/ethnicity, and
educational background.

\subsection{Procedure}\label{procedure}

An online version of this study was created through Qualtrics survey
software, and the survey link was distributed to participants on the via
SONA software through the University of Oregon Human Subjects pool.
After clicking the survey link, participants were informed of study
procedures and content through an informed consent process. After
completing the study, participants received academic credit as
compensation and were presented with debriefing materials. The
University of Oregon's s Office of Research Compliance (Institutional
Review Board) approved this study.

\subsection{Data analysis}\label{data-analysis}

We used R (Version 3.4.3; R Core Team, 2017) and the R-package
\emph{papaja} (Version 0.1.0.9842; Aust \& Barth, 2018) for all our
analyses. We calculated means and standard deviations for each of the
total scale scores. We also calculated Pearson's r correlation
coefficients between depressive symptoms, quality of life, and sleep
quality total scores. The authors interpreted our correlational results
as small (.10-.29), medium (.30-.49), and large correlations (.50-1.00).

\section{Results}\label{results}

\section{Discussion}\label{discussion}

\newpage

\section{References}\label{references}

\begingroup
\setlength{\parindent}{-0.5in} \setlength{\leftskip}{0.5in}

\hypertarget{refs}{}
\hypertarget{ref-R-papaja}{}
Aust, F., \& Barth, M. (2018). \emph{papaja: Create APA manuscripts with
R Markdown}. Retrieved from \url{https://github.com/crsh/papaja}

\hypertarget{ref-R-base}{}
R Core Team. (2017). \emph{R: A language and environment for statistical
computing}. Vienna, Austria: R Foundation for Statistical Computing.
Retrieved from \url{https://www.R-project.org/}

\endgroup


\end{document}
